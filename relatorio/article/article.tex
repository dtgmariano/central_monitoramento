\documentclass[a4paper,11pt]{article}
\usepackage[T1]{fontenc}
\usepackage[utf8]{inputenc}
\usepackage{lmodern}

\title{ICU Monitoring }

\author{A.N.Silva, D.T.G. Mariano, N.D. Linhares*, E.A.Lamounier**\\Biomedical Engineering Laboratory\\ Faculty of Electrical Engineering / Federal University of Uberlândia, Uberlândia-MG, Brasil\\andrei.ufu@gmail.com, dtgmariano@gmail.com, nicolailinhares@gmail.com}

\begin{document}

\maketitle
\tableofcontents

\begin{abstract}
The current article proposal is to introduce the development of an application of computer network within the hospital setting.
\end{abstract}

\section{Introduction}

  Technology is increasingly present in the medical field and aims to improve the care that is offered to patients in healthcare facilities. It also allows diagnoses to be faster and accurate, besides the option for less invasive treatments. On this scenario, a modern Intensive Care Units (ICU) has many equipments essential to support patient's life, such as multiparameter monitors, mechanical ventilators and infusion pumps, for instance. 
  Thus, it becomes a complex task to analyze all the data that is provided by these devices, especially considering the fact that the routine of doctors and nurses in the ICU is not just the care of patients, there are also administrative tasks to be carried out, for example. In addition, each patient may have a clinical picture different from the other, making the environment more complex and less predictable. All these variables are likely to complicate the care that is offered to patients. In this scenario, the idea of ​​creating a Central Monitoring in the ICU becomes interesting because it allows to concentrate the information of the various devices connected to each patient in a single room for one or more analysts may be on duty , checking and analyzing patterns and critical clinical inpatient in each location.
  The purpose of this project is to develop a central monitoring vital signs. Data will be obtained through network from each device and displayed on the screen , so that all patients can be observed in real time, simultaneously. 
  This center will also include alarm and warning capabilities, plus allowing statistical studies correlate the visualized data, and also try to predict possible anomalies in the cardiovascular system of patients according to the electrocardiographic signal ( ECG ) of the same . The center can also issue reports and advice to the medical staff , so that it may update with new information, make new analyzes of the clinical picture and can get the best strategies in the treatment of patients.
  The implementation of a core like this in a real environment of a UTI can bring benefits to the management of this environment , which as previously mentioned , is very complex and involves many variables that make it unpredictable. From the information extracted is possible to distribute the routines of nurses , for those patients whose condition proves more unstable to have a preferential service . Correlating the data enables a deeper analysis of patient outcomes in the ICU and alarm system to detect and forecast faster adverse events and reactions , which can ensure better preparation and greater flexibility in patient care.
  
\section{methodology}

\section{results}

\section{discussion}

\section{conclusion}

\section{acknowledgements}

\section{references}

\end{document}
